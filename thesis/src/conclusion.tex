\chapter{Conclusions}
\label{chap:conclusion}

Considering the specific ISP's topology, and the current end-to-end measurement
methodology, this dissertation proposes a data analytics framework to
detect and localize network events.
For such purpose, the mechanism tracks
statistical changes in the end-to-end QoS time series from different clients,
and correlates those patterns with traceroutes.
Besides, the proposed mechanism was compared with previous systems.
Finally, it were presented several outcomes when the procedure was applied to
real data.

The results show that, considering the exposed restrictions,
it is possible to only use end-to-end QoS metrics, and traceroutes, from
different clients, to identify and localize network events.
However, due to the lack of an events dataset, important
questions persist unanswered.
For instance, a quantitative accuracy study
would allow to check which types of events can't be handled by the
proposed procedure.
A dataset would also allow to better understand the events
patterns, how they impact the QoS metrics, and fine tune the system.

The use of end-to-end measurements has the advantage of dealing with
metrics that are directly related with the service perceived by the customers.
As an example, this feature can be used to rank simultaneous failure events
according with their impact to the customers.
Nonetheless, would be interesting to study the impact of introducing
internal network information to the framework.

Besides the data availability, the current measurement process imposed
several restrictions to this project.
In order to better explore the proposed solution,
a further continuation of this project
would require a stronger partnership with the ISP,
and also flexibility to adapt the measurement software.
However, this work can be used as a first guide to TGR's and ISP's engineers.

\section{Future Work}

Next is presented a list of future development directions of this work.
Considering the detection and localization of events, several adaptations to
the current measurement methodology are suggested.
Also, it is indicated deployment approaches which
were unable to be tackled by this project.

\begin{itemize}
\item
The used QoS metrics are affected by equipments in the path from the server to
the end-user, and also by devices in the reverse direction.
However, only the paths from end-users to servers are available.
Hence, a complete
localization analysis requires traceroutes from servers to end-users.
Besides, instead of only considering the round trip loss information,
the one way loss fraction from both directions can be tracked.
Further, the maximum achievable one way throughput measurement can be
implemented using UDP instead of TCP, which eliminates the interference of
performance degradation in the reverse path.

\item
If available, the HFC physical topology between the home router and the
first hop of the traceroute can be incorporated to the analysis.
Also, beyond only using end-to-end QoS metrics,
the proposed mechanism can be extended
to use internal network devices information, such as signal-to-noise ratio.

\item
A network failure events dataset can be built with ISP's auxiliary data.
As an example, customers complaints gathered from call centers can be used to,
during a specific time period, infer clients affected by a QoE deterioration,
which can then be translated to true network failure events.
Additionally, it is possible to use records from current failure detection
methods deployed by the ISP, such as manual inspection, or through equipments
that are able to self report specific faults.
However, through preliminary talks with ISP's engineers,
both databases are noisy, and mining useful information
from them can be a challenging task.
For instance, some devices flaws are manually
detected and corrected, but those informations aren't stored.
Also, during the dataset construction, the inferred events times can be
considerably different from their true occurrence time.
As stated in Chapter~\ref{chap:methodology}, a dataset could open new
supervised learning possibilities to the change point detection problem.
Besides, since the Tier-2 ISP is not a project partner, this kind of data from
it's infrastuture may not be available.

\item
Once a network events dataset is constructed, the correlation between
change points of different QoS metrics can reveal useful information
about the network behavior. This analysis was not done since the algorithms'
hyperparameters couldn't be optimized, hence, those comparisons could reach
wrong conclusions.

\item
A deeper knowledge of the Tier-2 ISP's infrastructure
can be used to model the Tier-2 network with finer granularity in the Spatial
Correlation procedure, which can improve the system's event localization
precision.

\item
It is planned in the ISP's roadmap to increase the number of tracked customers.
In this case, the system's computational performance can benefit from data
aggregation techniques, as it was done in Argus.
Besides, this increase will naturally improve the internal network equipments
coverage by the end-to-end measurements, which, as the previous topic, can
enhance the events localization precision.

\item
Once the system is deployed, the algorithms and parameters can be selected
through a reinforcement learning approach. If network operators
feedback the outcomes correctness, the system can adaptively optimize the used
strategies.

\item
Considering a real time processing environment, in order to decrease the
event detection delay, the system can adaptively control the measurement
frequency.
Increase the amount of data related to potentially problematic regions
can improve the system's output confidence in a short time period.
Also, it is possible to reduce the measurement frequency in well behaved
localities, which can lower the traffic overhead generated by measurements, and
increase the data analytics computational performance.

\item
Extend the mechanism to other types of network infrastructures.

\end{itemize}
