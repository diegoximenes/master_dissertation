\chapter{Conclusions}


Future work:
- collect datailed topology from the ISP which will enable more precise
localizations
- previous problems database and end-users reclamations to the ISP call center.
This can enable the problem be interpreted as a supervised problem.
- model the system as a reinforcement learning procedure, in which operators
can feedback the system with correct/mistakes in the detection and localization
of problems, then the system can be able to adapt automatically chossing the
best algorithms and hyperparameters.
- active increase mesuremet frequency in locations with potential problems,
avoiding incorrect classifications and decreasing the detection time.
- choose best network metrics that can increase the system performance
- if a more data is available try space aggregation techniques, as in Argus
- correlate path change in traceroute with changes in end-to-end metrics
- the proposed mechanism can also be expanded to detect attacks on networks
- is possible to detect and localize problems only with data collected
passively?
