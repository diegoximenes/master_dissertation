\chapter{Conclusions}
\label{chap:conclusion}

% Future work:
% \begin{itemize}
% \item collect datailed topology from the ISP which will enable more precise
% localizations
% \item previous problems database and end-users reclamations to the ISP call center.
% This can enable the problem be interpreted as a supervised problem.
% \item model the system as a reinforcement learning procedure, in which operators
% can feedback the system with correct/mistakes in the detection and localization
% of problems, then the system can be able to adapt automatically chossing the
% best algorithms and hyperparameters.
% \item active increase mesuremet frequency in locations with potential problems,
% avoiding incorrect classifications and decreasing the detection time.
% \item choose best network metrics that can increase the system performance
% \item if a more data is available try space aggregation techniques, as in Argus
% \item correlate path change in traceroute with changes in end-to-end metrics
% \item the proposed mechanism can also be expanded to detect attacks on networks
% \item catch other types of statistical changes, such as in periodicity
% \item is possible to detect and localize problems only with data collected
% \item collect on way metrics, including the traceroute, and change the TCP
% throughput to UDP to avoid interference of packet loss in the backward path
% passively?
% \item better explore the Tier-2 ISP topology and applied load balancing
% and MPLS techniques. This can improve the granularity of the user-groups
% associated with the Tier-2 network. Say that some paths are constant in the
% traceroute that traverse Tier-2 network. This can indicate parts of the
% network that doesn't applies load balancing, or only apply per flow load
% balancing, or the presence of a single minimum cost path. This info could be
% useful for topology inference.
% \item correlate routing updates with change points
% \end{itemize}
