\chapter{Literature Review - Networking Problem Localization Using End-to-End
Measurements}

Argus:
Detect and localize end-to-end service quality issues ISP's networks using
traffic data passively monitored at the ISP side, the ISP network topology,
routing tables, and geographic information. "Argus" has been successfully
deployed in a tier-1 ISP to monitor millions of users of its CDN service and
assist operators to detect and localize end-to-end service quality issues.

Active probing: periodically probe the service from agents at different network
locations to detect end-to-end performance issues.

Disadvantages of active probing:
- without active probes from a vast number of network locations throughout the
Internet, the monitoring coverage is limited and some end-to-end service
quality issues may not be detected.
- probe packets also place additional overhead on the network and may be
treated differently than normal packets.

Passive monitoring: each end-user detects the end-to-end service quality issues
individually based on performance metrics extracted from passively monitored
traffic and service quality issues detected by individual end-users are
correlated spatially and temporally to determine the scope of the problem.

Disadvantages of passive monitoring:
- effictiveness of these systems is limited by the sparcity of passive
end-to-end performance measurements for individual end-users, which further
depends how frequently they access the services. For example, if an end-user
only accesses the service a few times in a day, systems based on passive
monitoring at end-user side may not have sufficient samples to detect service
events.

Argus architecture:
Spatial aggregation -> temporal aggregation -> event detection -> event
localization -> event priorization

Spatial aggregation:
- Spatially aggregates end-users into user-groups, in order to avoid keeping
track of the end-to-end service quality associated with millions of individual
end-users. Each user-group is a set of end-users that share some common
attributes, such as BGP prefix or users in the same AS. These attributes can be
collected from different data sources such as network topology and routing
information. The type of spatial aggregation will influence the type of
location that is expected to localize problems.

Temporal aggregation:
- How to detect service anomaly events for each user-group? end-to-end
performace metrics from each user group can be quite noisy since they are
collected from different end-users. The Argus solution focus on the summary
statistics (e.g., 50th percentile, 95th percentile, min, max, etc) of the
distribution instead of based on individual end-to-end performance
measuremets. In this procedures some details about individual end-users are lost
but the goal is to detect service events that impact the user-groups.
For each user-group the measurements of all end-users of this group is
aggregated in time-bins, and then, for each time-bin, a summary statistics is
selected, forming then a summary time series. Different statistics may provide
an advatage for tracking certain type of issues. For example, the min may
capture the baseline RTT due to propagation delay while average can capture
network congestion. Argus uses median since they find median effective in
tracking service or network side issues while being robust to variablity in
performance of individual end-users due to their local processing or local
queuing delays.

Event detection:
Apply time series analysis techinques to extract service anomaly detection
algorithms. Due to scale of the system, it is desirable to have online
anomaly detection with minimal runtime complexity and memory requirements.
Argus applies additive Holt-Winters to do this detection. Argus also applies
some other techniques to improve robustness, for example, when there is a level
shift in the time series.

Event localization:
The localization algorithm is not presented in the paper.

Event prioritization:
The event prioritization occurs based on the significance of the anomaly
detected, measured through a score resulted from the holt winters, and also
considers the number of end-users impacted by the anomaly.

Results:
Argus was applied to RTT measurements in a CDN hosted in a tier-1 ISP. During a
one month period using time-bins of 1 hour. In this perior Argus detected 2909
anomaly events, and in general, lower level user-groups were more responsible
for these anomalies than the higher level groups. For each type of user-group,
only a small fraction are responsible for the anomaly events. Majority of the
anomalies are very short in duration.


