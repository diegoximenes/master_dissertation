\chapter{Introduction}

To better understand the performance of its own network, a major
tier-3 Brazilian \gls*{isp}, in partnership with the \gls*{ufrj} and a startup
incubated at this university,
established a project to monitor the service provided to a subset
of its customers.

Considering the \gls*{isp}'s cable-television infrastructure,
which runs \gls*{docsis},
\gls*{qos} metrics are gathered through a software placed at home routers
connected to cable modems. This software is responsible to perform end-to-end
measurements against servers strategically located by the \gls*{isp}.

In this context, guided by the specific \gls*{isp}'s network's characteristics,
and the current measurement process,
this dissertation aims to check the viability of
only using end-to-end \gls*{qos} measures, and traceroutes,
to identify and localize network events. An event can be interpreted as a
behavioral change in a network equipment, that affects the quality of service
perceived by the end-users,
such as a router failure. The localization procedure defines a set of
feasible locations where the event could have happened.

For such purpose, this work proposes a data analytics framework, which is able
to track statistical changes in the \gls*{qos} time series of different
clients.
To detect and localize events, the mechanism correlates these modification
patterns with traceroutes.
In order to increase the system's performance,
this dissertation also indicates possible improvements in the current
measurement methodology.

\section{Contributions}

Considering the specific \gls*{isp}'s network topology,
and the already implemented
measurement process, next is listed this dissertation's main contributions.

\begin{itemize}
\item
A data analytics procedure, that only uses the available end-to-end \gls*{qos}
measurements, and traceroutes, to detect and localize network events.

\item
A list of possible improvements in the measurement methodology currently
employed by the startup, in order to enhance the proposed system's performance.

\end{itemize}

\section{Dissertation Outline}

Chapter~\ref{chap:literature_review} consists of a literature review, that
describes three previous systems that use
end-to-end measurements to localize network events.
To track statistical changes in the time series, the current
work deploys change point detection methods. Then,
Chapter~\ref{chap:change_point_detection} presents several of these algorithms,
including the developed strategies to better handle the end-to-end \gls*{qos} data
characteristics.
In Chapter~\ref{chap:methodology} is detailed the measurement methodology and
the proposed data analytics workflow.
Chapter~\ref{chap:results} presents several results when the proposed pipeline
was applied to real data.
Finally, Chapter~\ref{chap:conclusion} concludes the work and
points future directions.

The code developed to support this dissertation is available at
\url{https://bitbucket.org/diegoximenes/master_thesis}.
Unfortunately, due to a \gls*{nda},
the data used in this project can not be open sourced.
