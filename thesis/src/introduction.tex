\chapter{Introduction}

To better understand the performance of it's own broadband services, a major
Tier-3 Brazilian \gls*{isp}, in partnership with UFRJ and the startup TGR,
established a project to collect end-to-end QoS metrics from a subset
of it's customers.

Considering the \gls*{isp}'s cable-television infrastructure with DOCSIS,
those metrics are gathered through a software placed at home routers connected
to cable modems. This software is responsible to perform end-to-end
measurements from those home routers against servers strategically
located by the \gls*{isp}\@.

In this context, guided by the specific \gls*{isp}'s network topology
and the current measurement process,
this dissertation aims to check the viability of
only using end-to-end QoS measures, and traceroutes,
to identify and localize network events. An event can be interpreted as a
behavioral change in a network equipment that affect QoS metrics of end-users,
such as a router failure. The localization procedure defines a set of
feasible locations where an event could have happened.

For such purpose, this work proposes a data analytics framework, which is able
to track statistical changes in the QoS metrics time series of different
clients.
To detect and localize events, the mechanism correlates those change
patterns with traceroutes.
In order to increase the system's performance,
this dissertation also indicates possible improvements on the current
measurement methodology.

\section{Contributions}

Considering the specific \gls*{isp}'s network topology, and the already implemented
measurement process, next is presented contributions of this dissertation.

\begin{itemize}
\item
Study the viability of only using the available end-to-end QoS metrics and
traceroutes to detect and localize network events.

\item Propose a data analytics pipeline to identify and localize events.

\item Guide possible improvements on the current measurement methodology,
in order to enhance the proposed system's performance. The construction
of the measurement process was out the scope of this dissertation.

\item List a set of \gls*{isp}'s data that could be gathered to increase the
framework's performance.

\end{itemize}

\section{Dissertation Outline}

In Chapter~\ref{chap:literature_review} is made a literature review, presenting
three previous systems that use
end-to-end measurements to localize the cause of network events.
To track changes in statistical properties of time series, the current
work deploys change point detection algorithms. Then,
Chapter~\ref{chap:change_point_detection} presents several of those methods,
including the developed strategies to better handle the end-to-end QoS data
characteristics.
In Chapter~\ref{chap:methodology} is detailed the measurement methodology and
the proposed data analytics workflow.
Chapter~\ref{chap:results} presents several results of the proposed pipeline
applied to real data.
Finally, Chapter~\ref{chap:conclusion} concludes the work and
points future directions.
