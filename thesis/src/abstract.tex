To better understand the performance of it's own network, a major
Tier-3 Brazilian ISP, in partnership with UFRJ and the startup TGR,
established a project to monitor the service provided to a subset
of it's customers.

In this context, guided by the specific ISP's network's characteristics,
and the current measurement process,
this dissertation aims to check the viability of
only using end-to-end QoS measures, and traceroutes,
to identify and localize network events. An event can be interpreted as a
behavioral change in a network equipment, that affect the quality of service
perceived by the end-users,
such as a router failure. The localization procedure defines a set of
feasible locations where the event could have happened.

For such purpose, this work proposes a data analytics framework, which is able
to track statistical changes in the QoS time series of different
clients.
To detect and localize events, the mechanism correlates these modification
patterns with traceroutes.
In order to increase the system's performance,
this dissertation also indicates possible improvements in the current
measurement methodology.
