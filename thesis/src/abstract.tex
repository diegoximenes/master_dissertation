To better understand the performance of it's own broadband services, a major
Tier-3 Brazilian ISP, in partnership with UFRJ and the startup TGR,
established a project to collect end-to-end QoS metrics from a subset
of it's customers.

In this context, guided by the specific ISP's network topology
and the current measurement process,
this dissertation aims to check the viability of
only using end-to-end QoS measures, and traceroutes,
to identify and localize network events. An event can be interpreted as a
behavioral change in a network equipment that affect QoS metrics of end-users,
such as a router failure. The localization procedure defines a set of
feasible locations where an event could have happened.

For such purpose, this work proposes a data analytics framework, which is able
to track statistical changes in the QoS metrics time series of different
clients.
To detect and localize events, the mechanism correlates those change
patterns with traceroutes.
In order to increase the system's performance,
this dissertation also indicates possible improvements on the current
measurement methodology.
