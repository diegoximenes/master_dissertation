Disadvantages of active probing:
- without active probes from a vast number of network locations throughout the
Internet, the monitoring coverage is limited and some end-to-end service
quality issues may not be detected.
- probe packets also place additional overhead on the network and may be
treated differently than normal packets.

Passive monitoring: each end-user detects the end-to-end service quality issues
individually based on performance metrics extracted from passively monitored
traffic and service quality issues detected by individual end-users are
correlated spatially and temporally to determine the scope of the problem.

Disadvantages of passive monitoring:
- effictiveness of these systems is limited by the sparcity of passive
end-to-end performance measurements for individual end-users, which further
depends how frequently they access the services. For example, if an end-user
only accesses the service a few times in a day, systems based on passive
monitoring at end-user side may not have sufficient samples to detect service
events.
