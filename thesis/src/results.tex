\chapter{Results}
\label{chap:results}

As stated in Chapter~\ref{chap:methodology}, the available data doesn't allow
an accuracy study. Therefore this chapter presents illustrative examples of the
proposed pipeline outcomes. It is possible to visually note that these examples
probably represent a real network event, that is a true positive.
Also it is made a comparison between
change points detected in different QoS metrics. It is also presented an
example that don't follow the proposed restrictions.

Due to it's simplicity, it the results of this chapter were obtained only using
the offline sliding windows method. Also it is used time series with a median
filter.
The used parameters were different to each
QoS metric, and were manually selected checking the change point detection
performance in specific time series. DESCRIBE PARAMETERS.

Eight months of data, may 2016 to december 2016 was divided in batches of
10 days.
For each batch, the pipeline was applied to the three types of events, the
round trip loss fraction, RTT, and maximum achievable upstream throughput, and
for each server.

Considering the detection of a real network event as a positive case,
Section~\ref{sec:possible_true_positive_examples} presents some possible true
positive cases. Section~\ref{sec:possible_false_negative_examples}, shows
possible false negative cases. Since network events can be considered rare,
true negative cases are abundant and not presented here. Possible false
positives were not found, since it were selected conservattives change point
detection and filter parameters.

\section{Possible True Positive Examples}
\label{sec:possible_true_positive_examples}

\section{Possible False Negative Examples}
\label{sec:possible_false_negative_examples}

\section{Comparison Between Change Points in Different QoS Metrics}
\label{sec:comparison_between_change_points_in_different_qos_metrics}

\section{Events Without Traceable Location}
\label{sec:events_without_traceable_location}
