\chapter{Results}
\label{chap:results}

As stated in Chapter~\ref{chap:methodology}, it is not possible to realize an
extensive accuracy study due to the absence of an events dataset.
However, this chapter presents illustrative examples when the proposed pipeline
was applied to real data.

It was considered 8 months of measurements, from may to december 2016.
This data was then divided in batches of 10 days, and for each batch,
a complete analysis was executed.

Considering all batches, after the End-Users Filtering procedure,
the number of target clients varied from X to X, through the X servers.
Figure~\ref{fig:path_length_distribution} presents a histogram of the path
length, in number of edges, between zero indegree vertexes and the server
vertex. Figure~\ref{fig:indegree_distribution} show the indegree distribution
of the user-groups.

For all cases, the time series were preprocessed with a median filter.
The change points were detected through the optimization model described in
Chapter~\ref{chap:change_point_detection}.
For each QoS metric, the algorithms' hyperparameters were manually selected.
It was opted by conservative values, in order to avoid change points that,
through a visual inspection, may be arguably not related to a network event.
Those algorithms choices were guided by two facts.
First, through a empirical visual analysis,
it was verified their reasonable perfomance with real data.
Also, the impact of their hyperparameters can be easily interpreted, which is
an important feature to manual tuning.

The event detection can be interpreted as a binary classification problem, in
which events represent the positive class.
Then, Section~\ref{sec:possible_true_positive_examples} presents some
possible true positive cases.
Also, it is illustrated the reasoning behind the events localization process.
Section~\ref{sec:possible_false_negative_examples} shows some possible false
negative cases.
Since network events can be considered rare, true negative cases are abundant,
and therefore, can be discarded from the analysis.
Possible false positives were not detected, which can be explained by two
reasons. First, it were selected conservative hyperparameters, which naturally
decrease the false positive rate. Also, since the system reported
several events, it was not possible to manually check all of them.

\section{Possible True Positive Examples}
\label{sec:possible_true_positive_examples}

Figure~\ref{fig:ts_event_before_first_hop} shows the set of clients that belong
to a specific user-group modeled by zero indegree vertex.

The system identified a network event at time X, in which only subset of those
clients perceived the event. Therefore, by the estabilished suppositions, they
share a physical equipment before the first hop, that caused this event.

Table~\ref{table:statistics_before} summarizes the number of cases with
this kind of outcome. Inconclusive events weren't detected. The ``single
client''
column represent the events in which only one customer perceived it.

Figure~\ref{fig:subgraph_event_first_hop_onward}, presents a subgraph of the
user-groups structure for a specific server. Considering a specific network
event, the red vertexes represent user-groups in which all end-users detected
this event. Figure~\ref{fig:ts_event_prefix_detected} show a subset of client
that
belong, to vertex X. The other customers of this user-group also have the same
change point pattern, but were not presented due to big number of users.
Figure~\ref{fig:ts_event_prefix_didnt_detect} show a subset of users that
belong to X but don't belong to Y.

Since all vertexes that belong to X also belong to X, it is not possible to
verify in which location the event can be placed. From the histogram in
Figure~\ref{fig:indegree_distribution}, it is possible to note that the one
indegree vertexes, which characterize this situation, are usual.

Table~\ref{table:statistic_prefix} summarizes the number of outcomes in which
the event is located in a prefix of the path between a zero indegree vertex to
the server vertex.

WRITE ABOUT CORRELATION BETWEEN PATHS

\section{Possible False Negative Examples}
\label{sec:possible_false_negative_examples}

Figure~\ref{fig:false_negative} show the detected change points for of all
clients that belong to a specific vertex with zero indegree.

It is possible to visually, note that, all time series changed it pattern at
time X. However, the system was unable to detect this change on client X.
This exemplify that the supposition that the change point algorithm is able to
detect all clients affected by a network event or no client is detected is a
strong supposition. Perhaps, relaxing this restriction, to instead of only
considering all clients in a user-group with the same pattern, to consider a
big fraction with the same pattern, can reduce this problem. However, since the
current dataset is small, this approach was not done.

\section{Events Without Traceable Location}
\label{sec:events_without_traceable_location}
