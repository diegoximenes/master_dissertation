Com o objetivo de melhor entender o desempenho dos seus serviços de banda larga,
um grande Tier-3 ISP brasileiro, em parceria com a UFRJ e a startup TGR,
estabeleceram um projeto para, através de medidas fim-a-fim de QoS,
avaliar o desempenho do serviço prestado em um subconjunto dos seus clientes.

Nesse contexto, guiada pelas características da rede do ISP e do atual
processo de medição, esta dissertação se propõe a
avaliar a viabilidade de apenas utilizar métricas fim-a-fim de QoS, e
traceroutes, para identificar e localizar eventos de rede. Um evento pode ser
interpretado como uma mudança no comportamento de um equipamento, que por sua
vez afeta a qualidade de serviço percebido pelos usuários,
como por exemplo, um defeito em um roteador.
O procedimento de localização define um conjunto de possíveis
locais onde o evento pode ter acontecido.

Com esse propósito, este trabalho propõe um framework de análise de dados,
que permite explorar mudanças estatísticas nas séries temporais de QoS de
diferentes clientes. Para detectar e localizar eventos, o mecanismo
correlaciona esses padrões de alteração com traceroutes.
A fim de aumentar o
desempenho do sistema proposto, esta dissertação também indica possíveis
melhoras na atual metodologia de medição.
