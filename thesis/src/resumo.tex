Com o objetivo de melhor entender o desempenho de sua rede,
um grande Tier-3 ISP brasileiro, em parceria com a UFRJ e a startup TGR,
estabeleceram um projeto para, atrav\'es de medidas fim-a-fim de QoS,
monitorar o servi\c{c}o prestado em um subconjunto dos seus clientes.

Nesse contexto, guiada pelas caracter\'isticas da rede do ISP, e do atual
processo de medi\c{c}\~ao, esta disserta\c{c}\~ao se prop\~oe a
avaliar a viabilidade de apenas utilizar m\'etricas fim-a-fim de QoS, e
traceroutes, para identificar e localizar eventos de rede. Um evento pode ser
interpretado como uma mudan\c{c}a no comportamento de um equipamento, que por
sua vez afeta a qualidade de servi\c{c}o percebida pelos usu\'arios,
como por exemplo, um defeito em um roteador.
O procedimento de localiza\c{c}\~ao define um conjunto de poss\'iveis
locais onde o evento pode ter acontecido.

Com esse prop\'osito, este trabalho prop\~oe um framework de an\'alise de dados,
que permite explorar mudan\c{c}as estat\'isticas nas s\'eries temporais de QoS
de diferentes clientes. Para detectar e localizar eventos, o mecanismo
correlaciona esses padr\~oes de altera\c{c}\~ao com traceroutes.
A fim de aumentar o
desempenho do sistema proposto, esta disserta\c{c}\~ao tamb\'em indica
poss\'iveis melhorias na atual metodologia de medi\c{c}\~ao.
